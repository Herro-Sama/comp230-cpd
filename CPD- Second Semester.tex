% Please do not change the document class
\documentclass{scrartcl}

% Please do not change these packages
\usepackage[hidelinks]{hyperref}
\usepackage[none]{hyphenat}
\usepackage{setspace}
\doublespace

% You may add additional packages here
\usepackage{amsmath}

% Please include a clear, concise, and descriptive title
\title{CPD Report}

% Please do not change the subtitle
\subtitle{COMP230 - CPD Tasks}

% Please put your student number in the author field
\author{1507729}

\begin{document}

\maketitle

\section*{Introduction}

The second semester of my second year has been challenging and has certainly introduced me to many new challenges but also equipped me to face many of them and taught me many new ways of handling issues and approaching problems. I have found though that, on reflection my ability to work in groups hasn't been as strong as I would like finding myself often asking what tasks people want me to work on rather than knowing what tasks need to be completed to make a better game overall and increase player experience. This is one of the key things I want to improve on as I will outline below.

\section{Be More Active during the development phase rather than passively working through things.}

I want to have a much more active role during the development phase of my group game, rather than being passive and waiting to see what work needs to be done. I think that this will be an important skill to develop for future development, though I will also need to make sure that the work I'm doing is going to make the game better rather than just showing that I'm doing work. This should be directly linked to the amount of commits I make during the production phase of development and should make the crunch time spike look much smaller, if I am consistently doing work rather than at the start and end of production. I'm going to spend more time interacting with the various designers in the team and working more closely to ensure that I should always know what we are working on and what the game could and should need next. This shouldn't be very difficult, as we have very few programmers and so designers will have to try and cover that slack meaning I will be working very closely with them regardless. I plan to begin doing this for my next group project which will be starting over the summer, this should be pretty evident that I have achieved this once we get back during September.

\section{Make time to fail, and experiment with things that don't work and to not be afraid to do that.}

I want to be less afraid to try new things and experiment with new systems even if they don't work, just so I can become more confident in my own programming abilities and more familiar with different approaches to problems. This is something that I will have to do going forward as we have so few programmers on my team for next year, I will be measuring this in my ability to do general programming/blueprint tasks in Unreal Engine 4. This is something I will be developing over the summer during the prototyping phase, as I will need to be able to develop any number of gameplay features that the new game may need and that I won't be familiar in doing. Being outside of my comfort zone is something that I have had deal with for many of the other modules, and shouldn't be anything beyond what you would expect of a gameplay programmer. Building core gameplay programming skills will be very important going forward into the industry. I will be working on these skills over the next few months mainly from May until July, with my group settling on their final choice by August if not sooner.

\section{Take regularly scheduled breaks rather than just being distracted after working for large amounts of time.}

Take time to have breaks everyday so I have a designated distraction time so, I don't get distracted talking to fellow teammates about non-work related things. Having a planned out set-time each day should be quite easy to observe, and it should be something that across development people will be able to see. I will make my group aware of when I am taking a break from doing group work to have a break and refocus myself these breaks will not run over their allotted time to ensure that it's not just me taking arbitrary breaks. This shouldn't be too difficult, I plan to use a similar break system from when I worked at a contact centre of having a fifteen minute break for roughly every three to four hours of work that I have done. This is something I will be doing starting from when my new group begins working and I will begin from day one with them.

\section{Start my day earlier, so I can consistently arrive for early morning meetings.}

Be more consistent with arriving for earlier meetings, especially in very cold weather. This should be easy to see as I should have better attendance for stand-up meeting, etc. in general. I will trying to wake up significantly earlier for these meetings to try and ensure that I am awake and ready to leave, rather than only giving myself a few minutes to get up and ready each day. I think this will be very difficult to accomplish but will be very important for maximising the amount of time I can spend getting work done with my group for next year. The time frame will be properly tested once we arrive back in September and the weather starts to get colder but it should be clear if I have achieved this as I will be in more as a result.

\section{Don't allow others to distract me from doing my work.}

I need to become better at telling people I am busy working and that I can't talk to them right now. This could be difficult to measure but at the very least it should show, in how often I spend time speaking to people around the studio rather than working. The core focus is for me to be able to say no to people and instead spend more time doing my work, this isn't going to include team related assistance like resolving issues or blockers, but will include general chat and distractions from work. I will have to tell them I don't have time. This isn't too unreasonable to think of as something to do, and wasn't as big of an issue in first year but certainly is now a growing problem that I will need to deal with. This is something I plan on doing most likely from September onward and will tie in with the breaks I will be having to make sure that if people do want to speak to me they also have a set time to do so.

\section*{Conclusion}

To conclude the points I have mentioned above, I have realised I have become quite relaxed about how I work in the studio and that I need to be more focused on work. This is something I will need to do as final year will be incredibly important to get myself back into working under strict conditions and to a tight deadline. I believe that I can achieve the things I have mentioned above but only through deliberate and conscious effort on my part. I also plan to ask my team to support me in helping make sure I met these targets going forward and can deliver more consistent work of a higher quality.






\end{document}
